\documentclass{article}

\usepackage[english]{babel}
\usepackage{filecontents}
\usepackage[letterpaper,top=2cm,bottom=2cm,left=3cm,right=3cm,marginparwidth=1.75cm]{geometry}

\usepackage{amsmath}
\usepackage{csquotes}
\usepackage{graphicx}
\usepackage{amsthm}

\usepackage[colorlinks=true, allcolors=blue]{hyperref}
\begin{filecontents}{\jobname.bib}
@article{ROTEM1987575,
title = {Analysis of a distributed algorithm for extrema finding in a ring},
journal = {Journal of Parallel and Distributed Computing},
volume = {4},
number = {6},
pages = {575-591},
year = {1987},
issn = {0743-7315},
doi = {https://doi.org/10.1016/0743-7315(87)90031-1},
url = {https://www.sciencedirect.com/science/article/pii/0743731587900311},
author = {D. Rotem and E. Korach and N. Santoro},
abstract = {A new and more detailed analysis of the unidirectional algorithm of Chang and Roberts for distributed extrema finding on a ring is presented. This analysis shows that this simple algorithm, which is known to be average case optimal, compares very favorably with all the other known algorithms as it requires O(n log n) messages with probability tending to one. A bidirectional version of this algorithm is presented and is shown to dominate the unidirectional one in its average message complexity. Finally, both the unidirectional and the bidirectional algorithms are generalized to perform k selection in the ring, i.e., find the k largest labeled processors.}
}
\end{filecontents}

\usepackage{natbib}
\usepackage{cleveref}
\newtheorem{lemma}{Lemma}
\newtheorem{theorem}{Theorem}
\begin{document}
\section*{Leader election in an undirected ring}
Our goal is to elect a leader in an undirected ring with $n$ nodes and asynchronous first-in-first-out communication.  We assume that every node has a unique identifier, and that all identifiers are
totally ordered. Initially, no node needs any knowledge about the identities of the nodes at the
other ends of its incident edges. Each node recognizes the directions of its edges, but there is no global sense of orientation in the ring. 

Here we describe an algorithm presented by \cite{ROTEM1987575}, which achieves the above and uses at most $3 n\log_{3}n + O(n)$ messages in the worst case.
\subsection*{The algorithm}
\subsubsection*{Definitions and assumptions}
The algorithm will consist of electoral rounds.

We will distinguish 4 states a node can be in:
\begin{description}
\item[\textit{passive}] --- node does not participate in election anymore and only passes messages
\item[\textit{E-candidate}] --- node is taking part in this electoral round
\item[\textit{A-candidate}] --- node awaits feedback from its active neighbours
\item[\textit{Leader}] --- node is the leader
\end{description}
(node that is not "\textit{passive}" will be also regarded as "\textit{active}")

Messages sent will consist of pairs of "type" and "value". Values will be from the set of unique identifiers of the nodes participating, while type can be  either \textit{E-msg}, \textit{A-msg} or \textit{T-msg}.

By  \textit{left}, \textit{right} and  \textit{both}, we will denote directions in witch node sends or from which it receives a message.

By $i$ we will denote the node's unique identifier and by $n(i)$ the node whose unique identifier is $i$.

Upon initialization each node starts with status \textit{E-candidate}.
\subsubsection*{Procedures}
Each round starts with each \textit{E-candidate} sending message (\textit{E-msg},$i$) in both of its directions.

Afterwards, each \textit{E-candidate} awaits to receive a message from both of its directions.

Let's denote them as $(\textit{E-msg},vl)$ received from \textit{left} and $(\textit{E-msg},vr)$ received from \textit{right}.

Let $y=max(i,vl,vr)$.

If $y>i$, then $n(i)$ will send $(\textit{A-msg},y)$ to left if $y=vr$ or to right if $y=vl$.
Then regardless if any message was sent from $n(i)$, it will change its state to \textit{A-candidate}.

If $i=vl=vr$ then $n(i)$ is the only \textit{E-candidate} in the ring. It should become the \textit{Leader} and send \textit{T-msg} in one of its directions.

Each \textit{A-candidate} awaits approval from both of its active neighbours. Meaning, it waits for message $(\textit{A-msg},i)$ from both sides. Afterwards, it changes its status to \textit{E-candidate} to participate in the next round.

If \textit{A-candidate} receives $(\textit{E-msg},i)$, it should treat it as a negative feedback, it should become \textit{passive} and pass the received message.

\textit{passive} node passes every message, except for potential $(\textit{A-msg},i)$ which is no longer needed.

If it receives $(\textit{T-msg},val)$, it learns that $n(val)$ is the leader, passes the message and finishes.

\textit{Leader} finishes when it receives $(\textit{T-msg},i)$.

It shall be noted that receiving \textit{T-msg} can be used to obtain a common sense of direction in the ring.
\subsection*{Correctness}
Let $j=max\{i | n(i) \text{ participates in the algorithm}\}$. 

The voting and checking procedure imply that at each electoral round $n(j)$ will defeat both of its immediate active neighbours; hence the number of \textit{passive} nodes increases monotonically. At the same time $n(j)$ will never be defeated (receive negative feedback). This implies that at some round $n(j)$ will be the only active node and both of its \textit{E-msg} messages will return to it, making it the leader.

Having been elected $n(j)$ can send a termination order \textit{T-msg} around the ring along with its identifier to notify other nodes of its election.
\subsection*{Complexity analysis}
By $m_i$ let's denote set of nodes participating in $i$-th election (number of \textit{E-candidates}).

Let $r_j(n(i))$ where $n(i)\in m_j$ be the immediate neighbour to the left of $n(i)$ who is still active in $j$-th round, symmetrically we define $l_j(n(i)$.
\begin{lemma}\label{lemma1}
    $|m_j|\geq 3|m_{j+1}|$ $\forall j$
\end{lemma}
\begin{proof}
If a node $n(i)\in m_j$ survives the $j$-th round, then it means that it must have received positive acknowledgment from both $r_j(n(i)$ and $l_j(n(i)$.
Therefore, because of the voting procedure, negative feedback will be received by $l^{2}_{j}(n(i)$, $l_{j}(n(i)$, $r_{j}(n(i)$ and $r^{2}_{j}(n(i)$. Thus, for each surviving node, at least $2$ were eliminated. 
Hence, at most a third of the processors from $m_j$ survive the $j$-th round.
\end{proof}
\begin{lemma}\label{lemma2}
    number of electoral rounds $\leq \lceil\log_{3}n\rceil$
\end{lemma}
\begin{proof}
The proof follows from \Cref{lemma1}.
\end{proof}

We can now notice that each round every $2$ immediate active nodes will exchange at most $3$ messages. That is, $2$ \textit{E-msg} messages and no more than one \textit{A-msg}.

Hence, follows:
\begin{theorem}
$|$Messages$|\leq 3n[\log_{3}n+O(n)\approx 1.89n \log_{2}n+O(n)$
\end{theorem}
\bibliographystyle{plainnat}
\bibliography{\jobname} 
\end{document}